% Options for packages loaded elsewhere
\PassOptionsToPackage{unicode}{hyperref}
\PassOptionsToPackage{hyphens}{url}
%
\documentclass[
]{article}
\usepackage{amsmath,amssymb}
\usepackage{iftex}
\ifPDFTeX
  \usepackage[T1]{fontenc}
  \usepackage[utf8]{inputenc}
  \usepackage{textcomp} % provide euro and other symbols
\else % if luatex or xetex
  \usepackage{unicode-math} % this also loads fontspec
  \defaultfontfeatures{Scale=MatchLowercase}
  \defaultfontfeatures[\rmfamily]{Ligatures=TeX,Scale=1}
\fi
\usepackage{lmodern}
\ifPDFTeX\else
  % xetex/luatex font selection
\fi
% Use upquote if available, for straight quotes in verbatim environments
\IfFileExists{upquote.sty}{\usepackage{upquote}}{}
\IfFileExists{microtype.sty}{% use microtype if available
  \usepackage[]{microtype}
  \UseMicrotypeSet[protrusion]{basicmath} % disable protrusion for tt fonts
}{}
\makeatletter
\@ifundefined{KOMAClassName}{% if non-KOMA class
  \IfFileExists{parskip.sty}{%
    \usepackage{parskip}
  }{% else
    \setlength{\parindent}{0pt}
    \setlength{\parskip}{6pt plus 2pt minus 1pt}}
}{% if KOMA class
  \KOMAoptions{parskip=half}}
\makeatother
\usepackage{xcolor}
\usepackage[margin=1in]{geometry}
\usepackage{color}
\usepackage{fancyvrb}
\newcommand{\VerbBar}{|}
\newcommand{\VERB}{\Verb[commandchars=\\\{\}]}
\DefineVerbatimEnvironment{Highlighting}{Verbatim}{commandchars=\\\{\}}
% Add ',fontsize=\small' for more characters per line
\usepackage{framed}
\definecolor{shadecolor}{RGB}{248,248,248}
\newenvironment{Shaded}{\begin{snugshade}}{\end{snugshade}}
\newcommand{\AlertTok}[1]{\textcolor[rgb]{0.94,0.16,0.16}{#1}}
\newcommand{\AnnotationTok}[1]{\textcolor[rgb]{0.56,0.35,0.01}{\textbf{\textit{#1}}}}
\newcommand{\AttributeTok}[1]{\textcolor[rgb]{0.13,0.29,0.53}{#1}}
\newcommand{\BaseNTok}[1]{\textcolor[rgb]{0.00,0.00,0.81}{#1}}
\newcommand{\BuiltInTok}[1]{#1}
\newcommand{\CharTok}[1]{\textcolor[rgb]{0.31,0.60,0.02}{#1}}
\newcommand{\CommentTok}[1]{\textcolor[rgb]{0.56,0.35,0.01}{\textit{#1}}}
\newcommand{\CommentVarTok}[1]{\textcolor[rgb]{0.56,0.35,0.01}{\textbf{\textit{#1}}}}
\newcommand{\ConstantTok}[1]{\textcolor[rgb]{0.56,0.35,0.01}{#1}}
\newcommand{\ControlFlowTok}[1]{\textcolor[rgb]{0.13,0.29,0.53}{\textbf{#1}}}
\newcommand{\DataTypeTok}[1]{\textcolor[rgb]{0.13,0.29,0.53}{#1}}
\newcommand{\DecValTok}[1]{\textcolor[rgb]{0.00,0.00,0.81}{#1}}
\newcommand{\DocumentationTok}[1]{\textcolor[rgb]{0.56,0.35,0.01}{\textbf{\textit{#1}}}}
\newcommand{\ErrorTok}[1]{\textcolor[rgb]{0.64,0.00,0.00}{\textbf{#1}}}
\newcommand{\ExtensionTok}[1]{#1}
\newcommand{\FloatTok}[1]{\textcolor[rgb]{0.00,0.00,0.81}{#1}}
\newcommand{\FunctionTok}[1]{\textcolor[rgb]{0.13,0.29,0.53}{\textbf{#1}}}
\newcommand{\ImportTok}[1]{#1}
\newcommand{\InformationTok}[1]{\textcolor[rgb]{0.56,0.35,0.01}{\textbf{\textit{#1}}}}
\newcommand{\KeywordTok}[1]{\textcolor[rgb]{0.13,0.29,0.53}{\textbf{#1}}}
\newcommand{\NormalTok}[1]{#1}
\newcommand{\OperatorTok}[1]{\textcolor[rgb]{0.81,0.36,0.00}{\textbf{#1}}}
\newcommand{\OtherTok}[1]{\textcolor[rgb]{0.56,0.35,0.01}{#1}}
\newcommand{\PreprocessorTok}[1]{\textcolor[rgb]{0.56,0.35,0.01}{\textit{#1}}}
\newcommand{\RegionMarkerTok}[1]{#1}
\newcommand{\SpecialCharTok}[1]{\textcolor[rgb]{0.81,0.36,0.00}{\textbf{#1}}}
\newcommand{\SpecialStringTok}[1]{\textcolor[rgb]{0.31,0.60,0.02}{#1}}
\newcommand{\StringTok}[1]{\textcolor[rgb]{0.31,0.60,0.02}{#1}}
\newcommand{\VariableTok}[1]{\textcolor[rgb]{0.00,0.00,0.00}{#1}}
\newcommand{\VerbatimStringTok}[1]{\textcolor[rgb]{0.31,0.60,0.02}{#1}}
\newcommand{\WarningTok}[1]{\textcolor[rgb]{0.56,0.35,0.01}{\textbf{\textit{#1}}}}
\usepackage{graphicx}
\makeatletter
\def\maxwidth{\ifdim\Gin@nat@width>\linewidth\linewidth\else\Gin@nat@width\fi}
\def\maxheight{\ifdim\Gin@nat@height>\textheight\textheight\else\Gin@nat@height\fi}
\makeatother
% Scale images if necessary, so that they will not overflow the page
% margins by default, and it is still possible to overwrite the defaults
% using explicit options in \includegraphics[width, height, ...]{}
\setkeys{Gin}{width=\maxwidth,height=\maxheight,keepaspectratio}
% Set default figure placement to htbp
\makeatletter
\def\fps@figure{htbp}
\makeatother
\setlength{\emergencystretch}{3em} % prevent overfull lines
\providecommand{\tightlist}{%
  \setlength{\itemsep}{0pt}\setlength{\parskip}{0pt}}
\setcounter{secnumdepth}{-\maxdimen} % remove section numbering
\ifLuaTeX
  \usepackage{selnolig}  % disable illegal ligatures
\fi
\usepackage{bookmark}
\IfFileExists{xurl.sty}{\usepackage{xurl}}{} % add URL line breaks if available
\urlstyle{same}
\hypersetup{
  pdftitle={Homework Assignment II: Survey adjustments},
  pdfauthor={Enter your group name (Peter, Robin, Theresa)},
  hidelinks,
  pdfcreator={LaTeX via pandoc}}

\title{Homework Assignment II: Survey adjustments}
\usepackage{etoolbox}
\makeatletter
\providecommand{\subtitle}[1]{% add subtitle to \maketitle
  \apptocmd{\@title}{\par {\large #1 \par}}{}{}
}
\makeatother
\subtitle{Enter your student ID}
\author{Enter your group name (Peter, Robin, Theresa)}
\date{Enter submission date}

\begin{document}
\maketitle

\begin{Shaded}
\begin{Highlighting}[]
\NormalTok{knitr}\SpecialCharTok{::}\NormalTok{opts\_chunk}\SpecialCharTok{$}\FunctionTok{set}\NormalTok{(}\AttributeTok{echo =} \ConstantTok{TRUE}\NormalTok{)}
\end{Highlighting}
\end{Shaded}

Please write up your answers, code, and results using this Markdown file
as a template. Save the file as \texttt{HA2\_yourstudentID.Rmd}. When
everything works properly, knit the final version to html format and
submit both Rmd and html files to the submission folder Homework 2 which
you can find on ILIAS.

The deadline for submission is 29.01.2024, 23:55.

In our application using the \texttt{FPEsurvey} data, we observed that
our sample underrepresented voters of Le Pen and other right-wing
candidates, while it overrepresented Mélenchon voters. We poststratified
candidate \texttt{choice} to actual first-round election results in
order to predict the 2017 runoff between Macron and Le Pen.
Poststratification dramatically improved our prediction of the runoff.
Unfortunately, such powerful poststratification information is rarely
available. More often than not, we have to make do with standard
sociodemographic information. In this assignment, we will see how far we
get when we poststratify by sociodemographics, particularly age and
gender. The distribution of the first-round electorate by age group and
gender is taken from the Enquête sur la participation électorale 2017,
the equivalent to the Repräsentative Wahlstatistik in Germany (note:
make sure that you have downloaded the data in
\texttt{age\_sex\_totals.rds} to your working directory.):

\begin{Shaded}
\begin{Highlighting}[]
\NormalTok{age\_sex\_totals }\OtherTok{\textless{}{-}} \FunctionTok{readRDS}\NormalTok{(}\StringTok{"age\_sex\_totals.rds"}\NormalTok{)}
\NormalTok{age\_sex\_totals}
\end{Highlighting}
\end{Shaded}

\begin{verbatim}
##        sex
## age      female    male
##   18-24 1501616 1445647
##   25-34 2347022 2052570
##   35-44 2928444 2549121
##   45-54 3356592 3090051
##   55-64 3238400 3026732
##   65-74 2886214 2492961
##   >75   2413333 1756146
\end{verbatim}

Note that the total number of first-round voters here is lower than the
official figures (35,084,850 versus 36,054,394). It is not fully clear
to me where this discrepancy comes from. In any case, in what follows,
we will look at proportions rather than totals.

You can earn one credit per bullet point. The maximum number of credits
is 6. Grading key: 6 credits \(\rightarrow\) grade 1, 5 credits
\(\rightarrow 2\), 4 credits \(\rightarrow 3\), 3 credits
\(\rightarrow 4\), 2 or 1 credits \(\rightarrow 5\). You will get one
extra credit for correct Markdown syntax, package, and data handling
(bullet 0.).

\begin{enumerate}
\def\labelenumi{\arabic{enumi}.}
\setcounter{enumi}{-1}
\tightlist
\item
  Preparations: load necessary packages (\texttt{tidyverse},
  \texttt{survey}), set working directory, load the data
  (\texttt{FPEsurvey.rda}), clean the data as we did in the tutorial,
  specify the \texttt{multi\_design} object.
\end{enumerate}

\subsubsection{Load packages, set working
directory}\label{load-packages-set-working-directory}

\begin{Shaded}
\begin{Highlighting}[]
\NormalTok{pacman}\SpecialCharTok{::}\FunctionTok{p\_load}\NormalTok{(tidyverse,survey)}
\CommentTok{\#setwd("C:/Users/peter/Nextcloud/LEHRE/ISM2324/Tutorial")}
\end{Highlighting}
\end{Shaded}

\subsubsection{Load and data}\label{load-and-data}

\begin{Shaded}
\begin{Highlighting}[]
\FunctionTok{load}\NormalTok{(}\StringTok{"FPEsurvey.rda"}\NormalTok{)}
\end{Highlighting}
\end{Shaded}

\subsubsection{Data cleaning}\label{data-cleaning}

Assign value labels / candidate names (check candidate order in
questionnaire!), transform to factors:

\begin{Shaded}
\begin{Highlighting}[]
\NormalTok{cand\_names }\OtherTok{\textless{}{-}} \FunctionTok{c}\NormalTok{(}\StringTok{"Arthaud"}\NormalTok{,}\StringTok{"Poutou"}\NormalTok{,}\StringTok{"Melenchon"}\NormalTok{,}\StringTok{"Hamon"}\NormalTok{,}\StringTok{"Macron"}\NormalTok{,}\StringTok{"Fillon"}\NormalTok{,}\StringTok{"Dupont{-}Aignon"}\NormalTok{,}\StringTok{"Le Pen"}\NormalTok{,}\StringTok{"Cheminade"}\NormalTok{,}\StringTok{"Asselineau"}\NormalTok{,}\StringTok{"Lassalle"}\NormalTok{)}

\NormalTok{age\_groups }\OtherTok{\textless{}{-}} \FunctionTok{c}\NormalTok{(}\StringTok{"18{-}24"}\NormalTok{,}\StringTok{"25{-}34"}\NormalTok{,}\StringTok{"35{-}44"}\NormalTok{,}\StringTok{"45{-}54"}\NormalTok{,}\StringTok{"55{-}64"}\NormalTok{,}\StringTok{"65{-}74"}\NormalTok{,}\StringTok{"\textgreater{}75"}\NormalTok{)}

\NormalTok{resp\_data }\OtherTok{\textless{}{-}}\NormalTok{ resp\_data }\SpecialCharTok{\%\textgreater{}\%} \FunctionTok{mutate}\NormalTok{(}\AttributeTok{choice =} \FunctionTok{factor}\NormalTok{(choice, }\AttributeTok{levels =} \FunctionTok{seq}\NormalTok{(}\DecValTok{1}\NormalTok{,}\DecValTok{11}\NormalTok{,}\DecValTok{1}\NormalTok{), }\AttributeTok{labels =}\NormalTok{ cand\_names), }\AttributeTok{age =} \FunctionTok{factor}\NormalTok{(age, }\AttributeTok{levels =} \FunctionTok{seq}\NormalTok{(}\DecValTok{1}\NormalTok{,}\DecValTok{7}\NormalTok{,}\DecValTok{1}\NormalTok{), }\AttributeTok{labels =}\NormalTok{ age\_groups), }\AttributeTok{sex =} \FunctionTok{recode\_factor}\NormalTok{(sex, }\StringTok{\textasciigrave{}}\AttributeTok{1}\StringTok{\textasciigrave{}} \OtherTok{=} \StringTok{"female"}\NormalTok{, }\StringTok{\textasciigrave{}}\AttributeTok{2}\StringTok{\textasciigrave{}} \OtherTok{=} \StringTok{"male"}\NormalTok{))}
\end{Highlighting}
\end{Shaded}

Transform Don't knows (0) on candidate rating variables to NAs

\begin{Shaded}
\begin{Highlighting}[]
\NormalTok{resp\_data }\OtherTok{\textless{}{-}}\NormalTok{ resp\_data }\SpecialCharTok{\%\textgreater{}\%}
  \FunctionTok{mutate\_at}\NormalTok{(}\FunctionTok{vars}\NormalTok{(Arthaud}\SpecialCharTok{:}\NormalTok{Lassalle), }\SpecialCharTok{\textasciitilde{}}\FunctionTok{na\_if}\NormalTok{(., }\DecValTok{0}\NormalTok{))}
\end{Highlighting}
\end{Shaded}

Drop the 149 respondents which did not indicate whom they voted for in
round 1; they are of no use for poststratification and raking.

\begin{Shaded}
\begin{Highlighting}[]
\NormalTok{resp\_data }\OtherTok{\textless{}{-}}\NormalTok{ resp\_data }\SpecialCharTok{\%\textgreater{}\%}
  \FunctionTok{filter}\NormalTok{(}\SpecialCharTok{!}\FunctionTok{is.na}\NormalTok{(choice))}
\end{Highlighting}
\end{Shaded}

\subsubsection{Specify a survey design
object}\label{specify-a-survey-design-object}

\begin{Shaded}
\begin{Highlighting}[]
\NormalTok{multi\_design }\OtherTok{\textless{}{-}} \FunctionTok{svydesign}\NormalTok{(}\AttributeTok{id =} \SpecialCharTok{\textasciitilde{}}\NormalTok{id }\SpecialCharTok{+}\NormalTok{ resp\_id, }\AttributeTok{probs =} \SpecialCharTok{\textasciitilde{}}\NormalTok{first\_stage }\SpecialCharTok{+}\NormalTok{ second\_stage, }\AttributeTok{data =}\NormalTok{ resp\_data)}
\end{Highlighting}
\end{Shaded}

\begin{enumerate}
\def\labelenumi{\arabic{enumi}.}
\tightlist
\item
  Estimate the shares of voters in the 14 poststratification cells
  defined by gender and age group using the
  \texttt{multi\_design}object. (hint: use \texttt{svytable()} from the
  survey package to estimate totals; you can either set the argument
  \texttt{Ntotals} to 1 or use the function \texttt{proportions()} to
  get the shares).
\end{enumerate}

\begin{Shaded}
\begin{Highlighting}[]
\FunctionTok{proportions}\NormalTok{(}\FunctionTok{svytable}\NormalTok{(}\SpecialCharTok{\textasciitilde{}}\NormalTok{age }\SpecialCharTok{+}\NormalTok{ sex, }\AttributeTok{design =}\NormalTok{ multi\_design))}
\end{Highlighting}
\end{Shaded}

\begin{verbatim}
##        sex
## age         female       male
##   18-24 0.08533917 0.06491612
##   25-34 0.10940919 0.08023341
##   35-44 0.10649161 0.07804522
##   45-54 0.10576222 0.08315098
##   55-64 0.07950401 0.06637491
##   65-74 0.06564551 0.04595186
##   >75   0.01385850 0.01531729
\end{verbatim}

\begin{enumerate}
\def\labelenumi{\arabic{enumi}.}
\setcounter{enumi}{1}
\tightlist
\item
  Compare the results to the population figures above. Which groups are
  overrepresented in our survey, which are underrepresented?
\end{enumerate}

\begin{Shaded}
\begin{Highlighting}[]
\FunctionTok{proportions}\NormalTok{(age\_sex\_totals)}
\end{Highlighting}
\end{Shaded}

\begin{verbatim}
##        sex
## age         female       male
##   18-24 0.04279957 0.04120432
##   25-34 0.06689559 0.05850303
##   35-44 0.08346747 0.07265589
##   45-54 0.09567070 0.08807364
##   55-64 0.09230195 0.08626892
##   65-74 0.08226383 0.07105520
##   >75   0.06878562 0.05005427
\end{verbatim}

\begin{itemize}
\tightlist
\item
  We could also subtract the two tables to have a more direct comparison
\end{itemize}

\begin{Shaded}
\begin{Highlighting}[]
\FunctionTok{proportions}\NormalTok{(}\FunctionTok{svytable}\NormalTok{(}\SpecialCharTok{\textasciitilde{}}\NormalTok{age }\SpecialCharTok{+}\NormalTok{ sex, }\AttributeTok{design =}\NormalTok{ multi\_design)) }\SpecialCharTok{{-}} \FunctionTok{proportions}\NormalTok{(age\_sex\_totals)}
\end{Highlighting}
\end{Shaded}

\begin{verbatim}
##        sex
## age           female         male
##   18-24  0.042539601  0.023711804
##   25-34  0.042513600  0.021730372
##   35-44  0.023024140  0.005389329
##   45-54  0.010091513 -0.004922660
##   55-64 -0.012797941 -0.019894008
##   65-74 -0.016618313 -0.025103340
##   >75   -0.054927118 -0.034736980
\end{verbatim}

\begin{itemize}
\tightlist
\item
  Generally, younger age groups tended to be overrepresented relative to
  older groups, a tendency that is slightly more pronounced among female
  voters.
\end{itemize}

\begin{enumerate}
\def\labelenumi{\arabic{enumi}.}
\setcounter{enumi}{2}
\tightlist
\item
  Poststratify the sample by age and gender (hint: you need to get rid
  of the NAs on the \texttt{age} and \texttt{sex}variables first, then
  specify the \texttt{multi\_design} object again, then poststratify.).
\end{enumerate}

\begin{Shaded}
\begin{Highlighting}[]
\NormalTok{resp\_data }\OtherTok{\textless{}{-}}\NormalTok{ resp\_data }\SpecialCharTok{\%\textgreater{}\%}
  \FunctionTok{filter}\NormalTok{(}\FunctionTok{complete.cases}\NormalTok{(age,sex))}

\NormalTok{multi\_design }\OtherTok{\textless{}{-}} \FunctionTok{svydesign}\NormalTok{(}\AttributeTok{id =} \SpecialCharTok{\textasciitilde{}}\NormalTok{id }\SpecialCharTok{+}\NormalTok{ resp\_id, }\AttributeTok{probs =} \SpecialCharTok{\textasciitilde{}}\NormalTok{first\_stage }\SpecialCharTok{+}\NormalTok{ second\_stage, }\AttributeTok{data =}\NormalTok{ resp\_data)}

\NormalTok{post\_design2 }\OtherTok{\textless{}{-}} \FunctionTok{postStratify}\NormalTok{(}\AttributeTok{design =}\NormalTok{ multi\_design, }\AttributeTok{strata =} \SpecialCharTok{\textasciitilde{}}\NormalTok{age}\SpecialCharTok{+}\NormalTok{sex, }\AttributeTok{population =}\NormalTok{ age\_sex\_totals)  }
\end{Highlighting}
\end{Shaded}

\begin{enumerate}
\def\labelenumi{\arabic{enumi}.}
\setcounter{enumi}{3}
\tightlist
\item
  Now use the poststratified design to estimate the first-round vote
  shares.
\end{enumerate}

\begin{Shaded}
\begin{Highlighting}[]
\FunctionTok{svymean}\NormalTok{(}\SpecialCharTok{\textasciitilde{}}\NormalTok{choice, post\_design2)}
\end{Highlighting}
\end{Shaded}

\begin{verbatim}
##                          mean     SE
## choiceArthaud       0.0033040 0.0019
## choicePoutou        0.0105521 0.0032
## choiceMelenchon     0.2608375 0.0209
## choiceHamon         0.0860839 0.0119
## choiceMacron        0.2727511 0.0120
## choiceFillon        0.1818040 0.0166
## choiceDupont-Aignon 0.0393783 0.0061
## choiceLe Pen        0.1100385 0.0146
## choiceCheminade     0.0044516 0.0017
## choiceAsselineau    0.0133325 0.0026
## choiceLassalle      0.0174664 0.0050
\end{verbatim}

\begin{enumerate}
\def\labelenumi{\arabic{enumi}.}
\setcounter{enumi}{4}
\tightlist
\item
  Compare this with the official results and our initial estimates using
  the \texttt{multi\_design}object. Did poststratifying by age and sex
  improve our estimates?
\end{enumerate}

\subsubsection{\texorpdfstring{Official results (using the
\texttt{cand\_totals} data frame from last week's
tutorial)}{Official results (using the cand\_totals data frame from last week's tutorial)}}\label{official-results-using-the-cand_totals-data-frame-from-last-weeks-tutorial}

\begin{Shaded}
\begin{Highlighting}[]
\NormalTok{cand\_totals }\OtherTok{\textless{}{-}} \FunctionTok{data.frame}\NormalTok{(}\AttributeTok{choice =} \FunctionTok{c}\NormalTok{(}\StringTok{"Arthaud"}\NormalTok{,}\StringTok{"Poutou"}\NormalTok{,}\StringTok{"Melenchon"}\NormalTok{,}\StringTok{"Hamon"}\NormalTok{,}\StringTok{"Macron"}\NormalTok{,}\StringTok{"Fillon"}\NormalTok{,}\StringTok{"Dupont{-}Aignon"}\NormalTok{,}\StringTok{"Le Pen"}\NormalTok{,}\StringTok{"Cheminade"}\NormalTok{,}\StringTok{"Asselineau"}\NormalTok{,}\StringTok{"Lassalle"}\NormalTok{), }\AttributeTok{totals =} \FunctionTok{c}\NormalTok{(}\DecValTok{232384}\NormalTok{,}\DecValTok{394505}\NormalTok{,}\DecValTok{7059951}\NormalTok{,}\DecValTok{2291288}\NormalTok{,}\DecValTok{8656346}\NormalTok{,}\DecValTok{7212995}\NormalTok{,}\DecValTok{1695000}\NormalTok{,}\DecValTok{7678491}\NormalTok{,}\DecValTok{65586}\NormalTok{,}\DecValTok{332547}\NormalTok{,}\DecValTok{435301}\NormalTok{))}

\NormalTok{cand\_totals }\OtherTok{\textless{}{-}}\NormalTok{ cand\_totals }\SpecialCharTok{\%\textgreater{}\%}
  \FunctionTok{mutate}\NormalTok{(}\AttributeTok{shares =}\NormalTok{ totals}\SpecialCharTok{/}\FunctionTok{sum}\NormalTok{(totals))}

\NormalTok{cand\_totals}
\end{Highlighting}
\end{Shaded}

\begin{verbatim}
##           choice  totals      shares
## 1        Arthaud  232384 0.006445373
## 2         Poutou  394505 0.010941940
## 3      Melenchon 7059951 0.195813886
## 4          Hamon 2291288 0.063550867
## 5         Macron 8656346 0.240091291
## 6         Fillon 7212995 0.200058695
## 7  Dupont-Aignon 1695000 0.047012300
## 8         Le Pen 7678491 0.212969631
## 9      Cheminade   65586 0.001819085
## 10    Asselineau  332547 0.009223481
## 11      Lassalle  435301 0.012073452
\end{verbatim}

\begin{itemize}
\tightlist
\item
  Poststratification by age and sex did reduce the vote share Mélenchon
  by 2 pp, and did increase the vote share for Fillon by 3 pp, but no
  for Le Pen (just the opposite!).
\end{itemize}

\begin{enumerate}
\def\labelenumi{\arabic{enumi}.}
\setcounter{enumi}{5}
\tightlist
\item
  Poststratification might fail either because the covariates are weakly
  linked to survey participation or the survey variable of interest. Are
  there any age and / or gender differences in first-round candidate
  choice? Use the \texttt{svytable()} function and the
  \texttt{multi\_design} to look at preferences by age group and gender
  separately.
\end{enumerate}

\begin{Shaded}
\begin{Highlighting}[]
\FunctionTok{proportions}\NormalTok{(}\FunctionTok{svytable}\NormalTok{(}\SpecialCharTok{\textasciitilde{}}\NormalTok{choice}\SpecialCharTok{+}\NormalTok{sex, }\AttributeTok{design=}\NormalTok{multi\_design),}\DecValTok{2}\NormalTok{)}
\end{Highlighting}
\end{Shaded}

\begin{verbatim}
##                sex
## choice               female        male
##   Arthaud       0.005154639 0.001680672
##   Poutou        0.009020619 0.013445378
##   Melenchon     0.264175258 0.302521009
##   Hamon         0.103092784 0.067226891
##   Macron        0.270618557 0.272268907
##   Fillon        0.157216495 0.144537815
##   Dupont-Aignon 0.046391753 0.031932773
##   Le Pen        0.118556701 0.122689076
##   Cheminade     0.002577320 0.006722689
##   Asselineau    0.009020619 0.020168067
##   Lassalle      0.014175258 0.016806723
\end{verbatim}

\begin{Shaded}
\begin{Highlighting}[]
\FunctionTok{proportions}\NormalTok{(}\FunctionTok{svytable}\NormalTok{(}\SpecialCharTok{\textasciitilde{}}\NormalTok{choice}\SpecialCharTok{+}\NormalTok{age, }\AttributeTok{design=}\NormalTok{multi\_design),}\DecValTok{2}\NormalTok{)}
\end{Highlighting}
\end{Shaded}

\begin{verbatim}
##                age
## choice                18-24       25-34       35-44       45-54       55-64
##   Arthaud       0.004854369 0.000000000 0.007905138 0.000000000 0.010000000
##   Poutou        0.009708738 0.007692308 0.011857707 0.011583012 0.020000000
##   Melenchon     0.383495146 0.365384616 0.249011858 0.243243243 0.270000000
##   Hamon         0.092233010 0.080769231 0.098814229 0.088803089 0.100000000
##   Macron        0.223300971 0.242307692 0.324110672 0.254826255 0.310000000
##   Fillon        0.082524272 0.080769231 0.154150198 0.154440154 0.160000000
##   Dupont-Aignon 0.024271845 0.050000000 0.035573123 0.057915058 0.015000000
##   Le Pen        0.155339806 0.138461538 0.098814229 0.138996139 0.080000000
##   Cheminade     0.000000000 0.007692308 0.003952569 0.000000000 0.010000000
##   Asselineau    0.009708738 0.015384615 0.003952569 0.027027027 0.020000000
##   Lassalle      0.014563107 0.011538462 0.011857707 0.023166023 0.005000000
##                age
## choice                65-74         >75
##   Arthaud       0.000000000 0.000000000
##   Poutou        0.006535948 0.000000000
##   Melenchon     0.143790850 0.225000000
##   Hamon         0.052287582 0.100000000
##   Macron        0.287581700 0.225000000
##   Fillon        0.294117647 0.350000000
##   Dupont-Aignon 0.058823529 0.025000000
##   Le Pen        0.117647059 0.050000000
##   Cheminade     0.006535948 0.000000000
##   Asselineau    0.006535948 0.000000000
##   Lassalle      0.026143791 0.025000000
\end{verbatim}

\begin{itemize}
\tightlist
\item
  Steep (and opposite) age gradients in voting for Mélenchon and Fillon,
  not so much for other candidates. Gender patterns less clear. Clearly
  though, Le Pen is underestimated across gender and age. Selection was
  not at random conditional on age and gender.
\end{itemize}

\end{document}
